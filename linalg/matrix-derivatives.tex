\documentclass{article}
\usepackage{amsmath}
\usepackage{amssymb}
\usepackage{hyperref}
\hypersetup{
    colorlinks=true,
    linkcolor=blue,
    filecolor=magenta,
    urlcolor=blue,
}

\begin{document}

\title{MLB Linear Algebra Toolkit: Matrix Derivatives}
\author{Grace Shepard}

\maketitle

\begin{abstract}
This document describes how to take derivatives of matrices. It covers material from section 2.1 of the \href{https://see.stanford.edu/materials/aimlcs229/cs229-notes1.pdf}{Stanford CS229 class notes 1}.
\end{abstract}

%%%%%%%%%%%%%%%%%%%%%%%%%%%%%%%%%%%%%%%%%%%%%%%%%%%%%%%%%%%%%%%%%%%%%
\section{Gradient of a Function of a Matrix}
Suppose there is some function $f(A)$ (like trace) that maps $A \in \mathbb{R}^{mxn} \rightarrow \mathbb{R}$ \\
We can take the gradient of this function with respect to the matrix $A$\\
\begin{center}
\[
\nabla_A f(A) =
\begin{bmatrix}
\frac{\partial}{\partial A_{11}} f(A) & \dots & \frac{\partial}{\partial A_{1n}} f(A)\\
\dots & \dots & \dots \\    
\frac{\partial}{\partial A_{m1}} f(A) & \dots & \frac{\partial}{\partial A_{mn}} f(A)   
\end{bmatrix}
\]
\end{center}
Next we will use the trace function to help with taking derivatives.


\section{Helpful Matrix and Derivative Facts}

\begin{equation}
\text{tr}AB = \text{tr}BA
\end{equation}

\begin{equation}
\text{tr}A = \text{tr}A^T
\end{equation}

\begin{equation}
\text{tr}(A+B) = \text{tr}A +\text{tr}B
\end{equation}

\begin{equation}
\text{tr}(aA) = a\text{tr}(A)
\end{equation}
\\
\\
\begin{equation}
\nabla_A\text{tr}(AB) = B^T
\end{equation}

\begin{equation}
\nabla_{A^T}f(A) = {(\nabla_Af(a))}^T
\end{equation}

\begin{equation}
\nabla_{A}\text{tr}ABA^TC = CAB+C^TAB^T
\end{equation}

\begin{equation}
\nabla_{A}|A| = |A|{(A^{-1})}^T
\end{equation}


\end{document}
\grid
